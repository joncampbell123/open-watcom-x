\documentclass{article}

% Load essential packages
\usepackage{fontspec}       % Enables system fonts & Unicode support
\usepackage{graphicx}       % Allows image inclusion
\usepackage{listings}       % Code listings
\usepackage{xcolor}         % For syntax highlighting in listings
\usepackage{hyperref}       % Clickable links in PDF
\usepackage{luacode}        % Allows Lua scripting

% Set a modern font (optional, replace as needed)
\setmainfont{Latin Modern Roman} % or "Times New Roman"

% Configure code listings
\lstset{
  language=C,                 % Default to C (can be changed per listing)
  basicstyle=\ttfamily\footnotesize,
  keywordstyle=\color{blue},
  commentstyle=\color{gray},
  stringstyle=\color{red},
  numbers=left,
  numberstyle=\tiny,
  stepnumber=1,
  breaklines=true
}


\begin{document}

\title{Open Watcom Documentation Example}
\author{Open Watcom Contributors}
\date{\today}
\maketitle

\section{Introduction}
This is an example of a minimal LuaLaTeX document for Open Watcom documentation.

\section{Including a Code Listing}
Here is an example C function:

\begin{lstlisting}
#include <stdio.h>

void hello() {
    printf("Hello, Open Watcom!\n");
}
\end{lstlisting}

\section{Adding an Image}
Here is a sample image (make sure `screenshot.png' exists in the same folder):

\begin{figure}[h]
    \centering
    \includegraphics[width=0.7\textwidth]{screenshot.png}
    \caption{Sample Screenshot}
    \label{fig:screenshot}
\end{figure}

\section{Using Lua for Dynamic Content}
We can run Lua code inside this document:

\begin{luacode}
  function greet(name)
    tex.print("Hello, " .. name .. "!")
  end
\end{luacode}

\noindent Lua output: \directlua{greet("Open Watcom")}

\section{Hyperlinks}
This document is available at \href{https://www.openwatcom.org}{Open Watcom's website}.

\end{document}
