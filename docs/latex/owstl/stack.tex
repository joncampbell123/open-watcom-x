% !TeX root = owstl.tex

\chapter{Stack}

\section{Introduction}

\code{template <class Type, class Container = std::dequeue>}\\
\code{class std::stack}

This chapter describes the \code{std::stack} adaptor. It is called an adaptor
because it wraps around a real container (the Container template parameter) to
store objects. It just provides a different interface to the underlying
container. The \code{std::stack} adaptor lacks the \code{begin} and \code{end}
methods, so you can't use iterators or the standard algorithms with it.

\section{Status}

The default container is currently a vector as deque has yet to be written

All members are complete:

\begin{enumerate}
\item \code{explicit stack( const Container &x = Container( ) )}
\item \code{empty( ) const}
\item \code{size( ) const}
\item \code{top( )} and \code{top( ) const}
\item \code{push( const value_type & )}
\item \code{pop( )}
\item \code{_Sane( )}
\item Relational operators
\end{enumerate}

\section{Design Details}

The \code{_Sane} method is used to check if the stack is in a valid state. It
is an \OW\ extension. The method returns \code{true} if the stack is valid and
\code{false} otherwise. The method works by calling the \code{_Sane} method of
the underlying container.
 