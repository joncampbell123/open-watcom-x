% !TeX root = owstl.tex

\chapter{Deque}

\section{Introduction}

The class template \code{std::deque} provides a random access sequence with
amortized \(O(1)\) \code{push_back} and \code{push_front} operations.

\section{Status}

The basic functionality of \code{std::deque} has been implemented. This
includes the specialized deque operations and deque iterators. However, the
more "exotic" vector-like operations (insert and erase in the middle of the
sequence) have not yet been implemented. There has been essentially no user
feedback.

\section{Design Details}

\subsection{Overall Structure}

This implementation is based on a circular buffer. Like a vector, a deque
object allocates more memory than it actually uses. In other words, its
capacity may be greater than its size. However, unlike a vector the sequence
stored in a deque is allowed to wrap around in the buffer resulting in
non-contiguous storage. This means an operation such as \code{&deq[0] + n} may
result in a pointer that is invalid even if \code{n} is less than the deque's
size. This behavior is allowed by the standard. \note{Reference?}

A deque object maintains two indices. The \code{head_index} refers to the
location in the buffer where the first item is stored. The \code{tail_index}
refers to the location in the buffer just past (after possibly wrapping
around) where the last item is stored. When \code{head_index == tail_index}
the deque is empty. To avoid the potential ambiguity of this condition, the
buffer is reallocated just before it is full, when \code{deq_length + 1 ==
buf_length}, so that the condition \code{head_index == tail_index} never
occurs due to a full buffer. This makes implementing some of the deque
operations much easier.

For example, deque iterators are represented using a pointer to the deque
object and an index value that marks the iterator's current position in the
deque's buffer. If the iterator's index value equals \code{head_index} this
can only mean the iterator is at the beginning of the sequence. It never means
that the iterator is just past the end of the sequence. This disambiguation
makes implementing \code{operator<} and the other relational operators on
iterator much more straight forward.

The general organization and style of deque's implementation follows that of
the other buffered sequences, \code{std::vector} and \code{std::string}. This
consistency is intentional. It is intended to make the \code{std::deque} code
easier to understand. It also opens up some possibility that all the buffered
sequences might one day share code.

\subsection{Alternative Implementations}

In addition to the circular buffer implementation an alternative approach was
considered that uses contiguous storage. The idea was to store the deque's
contents in the middle of the buffer so that some free space would be
available on either end for fast \code{push_back} and \code{push_front}
operations. If the deque grows to the point where one of the buffer ends is
reached, the active contents of the deque might be recentered (if the
allocated space was not too large) or completely reallocated (if the allocated
space was almost full).

This contiguous storage approach allows deque to be more vector-like and might
promote code sharing between deque and vector. For example, a vector would be a
special kind of deque in this case. However, at the time of this writing it is
unclear how such an implementation would best decide between recentering and
reallocation. More analysis is necessary to understand the issues involved.

\section{Open Watcom Extensions}

Because of this implementation's use of a circular buffer it is not difficult
to provide \code{capacity} and \code{reserve} methods for deque even though
the standard does not require them. As with vector, the \code{reserve} method
causes a deque to set aside enough memory so that no additional allocations or
internal copies will be needed until at least the reserved size is reached.
