% !TeX root = owstl.tex

\chapter{Introduction}

\section{Overview}

The Open Watcom Standard Template Library (OWSTL) is an implementation of the
C++ standard library defined in ISO/IEC 14882. This document describes the
design of OWSTL. Each section describes an element of the library and
typically includes an overview of the design, design decisions made and the
reasoning behind them, problems encountered, and explanations of the solutions
to those problems. It is hoped that a peer review of the code and design
documentation will be undertaken at some stage and that questions raised or
resulting changes made will be documented here.

\section{Philosophy}

OWSTL is written entirely from scratch. It does not, for example, build upon
an old HP/SGI code base or any other previous library. When a new element is
added to OWSTL the matter should be researched by the author before they
commence coding. The commercial compiler \OW\ is based on, made a name for
itself by producing high quality, fast code. The intention is to produce a
high performance library to complement that. This means choosing and
experimenting with the best algorithms possible and implementing them with
care.

OWSTL also attempts to be highly readable to encourage code inspection and
study.  That will encourage new developers to maintain and improve the library
and will give the greatest advantage in the long term.

\section{Status}

OWSTL is currently under development. Many elements of the library have not
yet been implemented. The code is mainly templates and currently resides in
under the \filepath{hdr} project. In the future, non-template classes or
functions may be factored out of the template code and be built into the
static and dynamic libraries. The existing library code is in
\filepath{bld/plusplus/cpplib}.

For example, it should be possible to separate the rebalancing algorithms from
the red-black tree code as these just manipulate pointers. They don't really
need to know the contained type. Reasonably thorough regression tests can be
found in \filepath{plustest/owstl}. These should be updated in parallel with
new functionality or fixes made to the library itself. Some Benchmarks can be
found in \filepath{bench/owstl}.

\section{Implementor's Notes}

When updating OWSTL, do the following:
\begin{itemize}
\item Check out latest source.
\item Run the regression tests. If any are broken, fix them or open an issue.
\item Update the source.
\item Update the regression tests.
\item Update this document.
\item Update the user documents (e.g., on the Wiki, if applicable).
\item Create a pull request.
\end{itemize}
