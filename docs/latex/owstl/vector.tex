% !TeX root = owstl.tex

\chapter{Vector}

\section{Introduction}

The class template \code{std::vector} provides a dynamic array of objects with
a type given by the type parameter. Unlike \code{std::string} vectors can be
instantiated with non-POD types. This complicates the implementation of
\code{std::vector} considerably, as discussed below.

\section{Status}

Most of the required functionality has been implemented. Some of the methods
are not yet exception safe.

\section{Design Details}

The internal structure of \code{vector} is very similar to that of
\code{string}. Any enhancement or bug fix applied to either of these templates
should be reviewed for possible application to the other. Like a string, a
vector allocates more raw memory than it needs. This allows the logical size
of the vector to increase without necessarily requiring a reallocation of
memory. However, unlike a string, a vector can contain objects with user
defined copy constructors and user defined \code{operator=}. In addition,
copying and assigning objects in a vector might cause an exception to be
thrown. These details make implementing \code{vector} more difficult than
implementing \code{string}.

For example, consider a vector of size 100 with 200 units of memory allocated.
Now suppose that 10 new objects are inserted in the middle of this vector. The
10 objects at the end of the vector need to be copying onto the raw memory
just past the end using a copy constructor. However, the other objects that
are moved will be placed on top of existing objects and thus must be copied
with an \code{operator=}.

If an exception occurs while the new objects are being constructed, the
objects constructed so far can be destroyed and the vector can be left in an
unmodified state. However, if an exception occurs after the new objects have
been created but during the assignment of the remaining objects it is somewhat
unclear how to best proceed. If the new objects are destroyed data may be lost
since the original copies of those objects may have already been overwritten.
Yet trying to restore the vector to its initial state is probably unwise; if
an exception has occurred while copying objects around, further copying is
unlikely to be successful. There is little choice but to leave the vector in a
corrupted, partially modified state.
